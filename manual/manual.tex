\documentclass[12pt]{article}
\usepackage{amssymb}
\usepackage{amsmath}
\usepackage{float}
\usepackage{graphicx}
\usepackage{subfig}
\usepackage[margin=1in]{geometry}
\usepackage{placeins}
\usepackage{listings}
\usepackage{tabularx}
\usepackage{natbib}
\linespread{1.3}

\title{User Manual for Terrabal  Version 0.1 }
\author{Christopher Marsh.}

\begin{document}

\maketitle
\clearpage
\section{Overview}

This document is to serve as the users manual for the \textit{Terrabal} model. This name derives from the Latin word for Earth, and from `bal', short for balance (as this is a full energy and mass balance model). The fact that saying this name quickly sounds like 'terrible' is no mistake; it is important to remember that all models are wrong, and that some may be useful. It is hoped that this model falls into the wrong yet useful category.

The design principles for Terrabal are three-fold: 1) to maintain a modular system for interconnecting different hydrological process conceptualizations, 2) to allow for multi-scale model development, and 3) to provide a meanse to quickly and reproducibly compare model structure and algorithms. 

\section{Interpolation}

Interpolation of input meterological variables is done \textit{a priori} to the rest of the model running. 


\section{Modules}
Hydrological process conceptualizations are written into modules. Each module has a set of pre- and post-conditions that denote which variables are required and which are provided. Modules dependencies are then determined at runtime. Further, modules may be either 'element-parallel' or 'domain-parallel'. All element-parallel modules only require a single element with no dependency upon other elements, and can be batched together in a parallel pipeline, fullfilling dependency order. Domain-parallel requires elements to known other elements properties, and thus must be run separate. 




\section{Variable names}
In order to present a standar way of accessing variable names, a map of internal to external names is used. These are case sensitive.

\begin{center}
\begin{tabular}{c|c}
	\textbf{Desc}  &  \textbf{Internal model name}  \\
	Relative Humidity & RH \\
	Air temperature & Tair \\
	Timestep & timestep \\
	Angle between slope normal and sun & solar_S_angle \\

\end{tabular}
\end{center}

\clearpage
\bibliographystyle{agu08}
\bibliography{refs}

\end{document}

